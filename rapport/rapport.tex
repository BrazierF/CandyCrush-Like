   %%%%%%%%%%%%%%%%%%%%%%%%%%%%%%%%%%%%%%%%%
% University Assignment Title Page 
% LaTeX Template
% Version 1.0 (27/12/12)
%
% This template has been downloaded from:
% http://www.LaTeXTemplates.com
%
% Original author:
% WikiBooks (http://en.wikibooks.org/wiki/LaTeX/Title_Creation)
%
% License:
% CC BY-NC-SA 3.0 (http://creativecommons.org/licenses/by-nc-sa/3.0/)
% 
% Instructions for using this template:
% This title page is capable of being compiled as is. This is not useful for 
% including it in another document. To do this, you have two options: 
%
% 1) Copy/paste everything between \begin{document} and \end{document} 
% starting at \begin{titlepage} and paste this into another LaTeX file where you 
% want your title page.
% OR
% 2) Remove everything outside the \begin{titlepage} and \end{titlepage} and 
% move this file to the same directory as the LaTeX file you wish to add it to. 
% Then add \input{./title_page_1.tex} to your LaTeX file where you want your
% title page.
%
%%%%%%%%%%%%%%%%%%%%%%%%%%%%%%%%%%%%%%%%%
%\title{Title page with logo}
%----------------------------------------------------------------------------------------
%   PACKAGES AND OTHER DOCUMENT CONFIGURATIONS
%----------------------------------------------------------------------------------------

\documentclass[12pt]{article}
\usepackage[francais]{babel}
\usepackage[utf8]{inputenc}
\usepackage{amsmath}
\usepackage{graphicx}
\usepackage[colorinlistoftodos]{todonotes}

\begin{document}

\begin{titlepage}

\newcommand{\HRule}{\rule{\linewidth}{0.5mm}} % Defines a new command for the horizontal lines, change thickness here

\center % Center everything on the page
 
%----------------------------------------------------------------------------------------
%   HEADING SECTIONS
%----------------------------------------------------------------------------------------

\textsc{\LARGE  \'{E}cole Polytechnique de Montr\'{e}al, Qu\'{e}bec, Canada}\\[1.5cm] % Name of your university/college
\textsc{\Large INF8405 - Informatique mobile
}\\[0.5cm] % Major heading such as course name
\textsc{\large Travail Pratique N.1}\\[0.5cm] % Minor heading such as course title

%----------------------------------------------------------------------------------------
%   TITLE SECTION
%----------------------------------------------------------------------------------------

\HRule \\[0.4cm]
{ \huge \bfseries Application de jeu pour Android}\\[0.4cm] % Title of your document
\HRule \\[1.5cm]
 
%----------------------------------------------------------------------------------------
%   AUTHOR SECTION
%----------------------------------------------------------------------------------------

\begin{minipage}{0.5\textwidth}
\begin{flushleft} \large
\emph{Auteurs:}\\
Franck \textsc{Brazier}  1815797\\ % Your name 
Abbas  \textsc{Omidali} 1759476\\ % Your name 
Farshad \textsc{Tir} 1769679 \\ % Your name 
\end{flushleft}
\end{minipage}
~
\begin{minipage}{0.4\textwidth}
\begin{flushright} \large
\emph{Soumis à :} \\
Fabien  \textsc{BERQUEZ} % Supervisor's Name
\end{flushright}
\end{minipage}\\[2cm]

% If you don't want a supervisor, uncomment the two lines below and remove the section above
%\Large \emph{Author:}\\
%John \textsc{Smith}\\[3cm] % Your name


%----------------------------------------------------------------------------------------
%   LOGO SECTION
%----------------------------------------------------------------------------------------

\includegraphics[height=3.8cm]{poly.jpg} % Include a department/university logo - this will require the graphicx package
 

%----------------------------------------------------------------------------------------
%   DATE SECTION
%----------------------------------------------------------------------------------------

{\large \today}\\[2cm] % Date, change the \today to a set date if you want to be precise

%----------------------------------------------------------------------------------------

\vfill % Fill the rest of the page with whitespace

\end{titlepage}

\section{Introduction}
Le but de de ce TP est de créer une application pour le jeu match-3. Nous avons voulu pour la phase de jeu proposer un modèle Modèle-Vue-Contrôleur ce qui nous semblait le plus approprié.
\section{Présentation technique}
\subsection{Activités}
Notre application possède deux types d'activités, des activités où ne sont présent que des boutons pour pouvoir accéder à l'activité de jeu et les activités de jeu et de règles qui sont les activités finales. Ces activités intermédiaires sont :
\begin{itemize}
\item Main Activity \\
Cette activité est l'activité première de l'application et permet d'accéder à la page de choix des niveaux,  aux règles du jeu ou  le quitter
\item StartActivity \\
Cette activité permet d'effectuer le choix du niveau auquel l'utilisateur veut jouer, l'utilisateur est bloqué si le niveau précédent n'est pas réussi
\end{itemize}
Les activités dites finales sont : 
\begin{itemize}
\item RulesActivity \\
Cette activité comprend uniquement les règles du jeu (compris dans des \textit{TextViews}) sans autre intéraction avec l'utilisateur
\item GameActivity \\
L'activité la plus importante de l'application qui est la \textit{vue} du jeu, elle récupère la bonne  grille depuis un fichier xml grâce à un extra dans la création de l'activité et initialise le contrôleur de jeu.
\end{itemize}
Chaque classe hérite de la classe BaseActivity qui implémente les fonctions communes à chaque activité, par exemple celles appelées lorsque l'on appuie sur un bouton de la bar d'application située en haut de l'écran.


\subsection{Classes}
Notre application possède 5 classes essentielles pour le fonctionnement du jeu. il y a deux classes qui implémentent des Listeners. (\textit{OnTouch} et \textit{OnDrag}). L'\textit{OnTouchListener} permet de faire disparaître l'élement lorsque l'on appuie dessus et qu'on comment à le déplacer.   L'\textit{OnDragListener} a surtout pour but de détecter les éléments que l'utilisateur veut échanger et d'\textit{envoyer} cette requête au contrôleur. Les autres classes sont Circle, Level et LevelController.
\subsubsection{Level}
L'instance de cette classe contient tous les caractéristiques (nombre de lignes, colonnes, score à atteindre,...) d'un niveau de jeu.
\subsubsection{LevelController}
Certains de ses attributs sont statique pour pouvoir être accessible tout le long de l'exécution comme le niveau maximal débloqué.
LevelController est responsable de la grille, de permettre un échange valide, de supprimer les éléments alignés et les remplacer.

\subsubsection{Circle}
\textit{Circle} est la représentation d'un élément, ses attributs sont donc son positionnement, sa couleur, nous avons aussi rajouté la position cible de l'élément (par exemple lors d'un échange orchestré par l'utilisateur, la position cible sera celle de l'autre élément, cette position peut ne pas être valable pour un échange) 
\section{Difficultés particulières}
Nous avons éprouvé des difficultés essentiellement pour le contrôleur, spécialement pour créer un état temporaire pour voir s'il y a des alignements et si possible annuler cet état et être dans le précédent sans laisser de trace de cet état. \\
Nous avons perdu du temps pour pouvoir changer la taille des éléments, cependant nous avons trouvé une solution qui n'est pas optimale en mettant la taille dans des fichier xml en fonction de la taille de l'écran. Notre problème venait du fait que nos éléments héritait du boutons et qu'il était difficile de pouvoir modifier leur taille, nous avons donc fait hérité de \textit{TextView} nos éléments  

\section{Critiques et suggestions}
Nous avons identifié quelques points faibles dans notre application. Voici la liste : 
\begin{itemize}
\item Nous avons pas réussi à faire en sorte que la grille et les éléments s'épousent parfaitement avec la taille de l'écran
\item Nous avons essayé après coup de créer un mode pour voir les différentes étapes dans un même coup ou échange.  
\item Des artefacts peuvent apparaître dans le jeu (des alignements qui devraient disparaître) ou un élément qui reste dans son état \textit{à enlever} identifiable par sa transparence.
\item Nous aurions aimer pouvoir créer des animations.
\item Des textes plus plaisant à lire. Le but serait d'avoir un meilleur agencement des \textit{TextViews}.
\item Une amélioration serait de pouvoir créer des niveaux aléatoirement avec leur difficultés. 
\item Nous avons utilisé nos propres smartphones pour les tests, nous avons seulement pu essayer d'obtenir un aperçu sur les plus grands appareil grâce à la fonctionnalité \textit{Preview} d' \textit{Android Studio} 
\end{itemize}

\section{Autres}
Pour résoudre nos différents problèmes ou manque de connaissances en Android nous avons consulté plusieurs sites, la liste \textbf{non-exhaustive} de ces sites sont présent dans le fichier \textit{src/doc/sources}. Ces sites ont inspiré notre code et des parties de celui-ci sont directement issues de ces sites.
\end{document}