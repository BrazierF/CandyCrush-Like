   %%%%%%%%%%%%%%%%%%%%%%%%%%%%%%%%%%%%%%%%%
% University Assignment Title Page 
% LaTeX Template
% Version 1.0 (27/12/12)
%
% This template has been downloaded from:
% http://www.LaTeXTemplates.com
%
% Original author:
% WikiBooks (http://en.wikibooks.org/wiki/LaTeX/Title_Creation)
%
% License:
% CC BY-NC-SA 3.0 (http://creativecommons.org/licenses/by-nc-sa/3.0/)
% 
% Instructions for using this template:
% This title page is capable of being compiled as is. This is not useful for 
% including it in another document. To do this, you have two options: 
%
% 1) Copy/paste everything between \begin{document} and \end{document} 
% starting at \begin{titlepage} and paste this into another LaTeX file where you 
% want your title page.
% OR
% 2) Remove everything outside the \begin{titlepage} and \end{titlepage} and 
% move this file to the same directory as the LaTeX file you wish to add it to. 
% Then add \input{./title_page_1.tex} to your LaTeX file where you want your
% title page.
%
%%%%%%%%%%%%%%%%%%%%%%%%%%%%%%%%%%%%%%%%%
%\title{Title page with logo}
%----------------------------------------------------------------------------------------
%   PACKAGES AND OTHER DOCUMENT CONFIGURATIONS
%----------------------------------------------------------------------------------------

\documentclass[12pt]{article}
\usepackage[francais]{babel}
\usepackage[utf8]{inputenc}
\usepackage{amsmath}
\usepackage{graphicx}
\usepackage[colorinlistoftodos]{todonotes}

\begin{document}

\begin{titlepage}

\newcommand{\HRule}{\rule{\linewidth}{0.5mm}} % Defines a new command for the horizontal lines, change thickness here

\center % Center everything on the page
 
%----------------------------------------------------------------------------------------
%   HEADING SECTIONS
%----------------------------------------------------------------------------------------

\textsc{\LARGE  \'{E}cole Polytechnique de Montr\'{e}al, Qu\'{e}bec, Canada}\\[1.5cm] % Name of your university/college
\textsc{\Large INF8405 - Informatique mobile
}\\[0.5cm] % Major heading such as course name
\textsc{\large Travail Pratique N.1}\\[0.5cm] % Minor heading such as course title

%----------------------------------------------------------------------------------------
%   TITLE SECTION
%----------------------------------------------------------------------------------------

\HRule \\[0.4cm]
{ \huge \bfseries Application de jeu pour Android}\\[0.4cm] % Title of your document
\HRule \\[1.5cm]
 
%----------------------------------------------------------------------------------------
%   AUTHOR SECTION
%----------------------------------------------------------------------------------------

\begin{minipage}{0.5\textwidth}
\begin{flushleft} \large
\emph{Auteurs:}\\
Franck \textsc{Brazier}  1815797\\ % Your name 
Abbas  \textsc{Omidali} 1759476\\ % Your name 
Farshad \textsc{Tir} 1769679 \\ % Your name 
\end{flushleft}
\end{minipage}
~
\begin{minipage}{0.4\textwidth}
\begin{flushright} \large
\emph{Soumis à :} \\
Fabien  \textsc{BERQUEZ} % Supervisor's Name
\end{flushright}
\end{minipage}\\[2cm]

% If you don't want a supervisor, uncomment the two lines below and remove the section above
%\Large \emph{Author:}\\
%John \textsc{Smith}\\[3cm] % Your name


%----------------------------------------------------------------------------------------
%   LOGO SECTION
%----------------------------------------------------------------------------------------

\includegraphics[height=3.8cm]{poly.jpg} % Include a department/university logo - this will require the graphicx package
 

%----------------------------------------------------------------------------------------
%   DATE SECTION
%----------------------------------------------------------------------------------------

{\large \today}\\[2cm] % Date, change the \today to a set date if you want to be precise

%----------------------------------------------------------------------------------------

\vfill % Fill the rest of the page with whitespace

\end{titlepage}

\section{Introduction}

\section{Présentation technique}7
\subsection{Activités}
Notre applications possèdent deux types d'activités, des activités où ne sont présent que des boutons pour pouvoir accéder à l'activité de jeu et les activités de jeu et de règles qui sont les activités finales. Ces activités intermédiaires sont :
\begin{itemize}
\item Main Activity \\
Cette activité est l'activité première de l'application et permet d'accéder à la page de choix des niveaux,  aux règles du jeu ou  le quitter
\item StartActivity \\
Cette activité permet d'effectuer le choix du niveau auquel l'utilisateur veut jouer, l'utilisateur est bloqué si le niveau précédent n'est pas réussi
\end{itemize}
Les activités finales sont : \\
\begin{itemize}
\item RulesActivity \\
Cette activité comprend uniquement les règles du jeu (compris dans des \textit{TextViews}) sans autre intéraction avec l'utilisateur
\item GameActivity \\
L'activité la plus importante de l'application qui est la \textit{vue} du jeu, elle récupère la bonne  grille depuis un fichier xml grâce à un extra dans la création de l'activité et initialise le contrôleur de jeu
\end{itemize}
Chaque classe hérite de la classe BaseActivity qui implémente les fonctions communes à chaque activité, par exemple celles appelées lorsque l'on appuie sur un bouton de la bar d'application située en haut de l'écran.


\subsection{Classes}
Notre application possède 5 classes, essentiel pour le fonctionnement du jeu. il y a deux classes qui implémentent des Listeners. (\textit{OnTouch} et \textit{OnDrag}). L'\textit{OnTouchListener} permet de faire disparaître l'élement lorsque l'on appuie dessus et qu'on comment à le déplacer.   L'\textit{OnDragListener} a surtout pour but de détecter les éléments que l'utilisateur veut échanger et d'\textit{envoyer} cette requête au contrôleur.
 

\section{Difficultés particulières}

\section{Critiques et suggestions}

\end{document}